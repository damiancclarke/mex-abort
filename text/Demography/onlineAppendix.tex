\documentclass[a4paper, 11pt]{article}

%-------------------------------------------------------------------------------
%--- (a) Required Packages
%-------------------------------------------------------------------------------
\usepackage{amsmath,amsfonts,amssymb,amsthm}
\usepackage{adjustbox}
\usepackage{mwe}
\usepackage{authblk}
\usepackage[english]{babel}
\usepackage[para,online,flushleft]{threeparttable}
\usepackage{lscape}
\usepackage{threeparttablex}
\usepackage{tabu}
\usepackage{booktabs}
\usepackage{caption}
\usepackage{subcaption}
\usepackage[usenames, dvipsnames]{color}
\usepackage{epstopdf}
\usepackage[capposition=top]{floatrow}
\usepackage{framed}
%\usepackage[T1]{fontenc}
\usepackage{lmodern}
\usepackage{graphicx}
\usepackage{hyperref}
\usepackage[utf8]{inputenc}
\usepackage{lscape}
\usepackage{multirow}
\usepackage{natbib}
\usepackage{setspace}
\usepackage{rotating}
\usepackage{subcaption}
\usepackage{subfloat}
\usepackage{url}
\usepackage{wrapfig}
\usepackage{multicol}
\usepackage[toc,page]{appendix}
\usepackage{float}
\interfootnotelinepenalty=10000
\usepackage[a4paper]{geometry}
\usepackage{textcomp}
\usepackage{longtable}
\usepackage{babel,blindtext}
%-------------------------------------------------------------------------------
%--- (b) Specific margins
%-------------------------------------------------------------------------------
\setlength  \textwidth{\paperwidth}
\setlength  \textheight{\paperheight}
\setlength  \oddsidemargin{3cm}
\setlength  \topmargin{-1.2cm}
\setlength  \footnotesep{2ex}
\addtolength\textheight{-6cm}
\addtolength\textwidth{-6cm}
\addtolength\oddsidemargin{-1in}
 
%%ADDED SPACE BETWEEN PARAGRAPHS 
%-------------------------------------------------------------------------------
%--- (c) Internal Ref stype
%-------------------------------------------------------------------------------

\hypersetup{
    colorlinks=true,   
    linkcolor=Black,
    citecolor=BlueViolet,
    filecolor=BlueViolet,
    urlcolor=Black
}

 
%-------------------------------------------------------------------------------
%---  ABSTACT
%-------------------------------------------------------------------------------
\renewcommand\thesection{\Alph{section}}
\renewcommand*{\thepage}{A\arabic{page}}



\renewcommand{\thetable}{A\arabic{table}}
\renewcommand{\thefigure}{A\arabic{figure}}
\setcounter{page}{1}

\begin{document}
\begin{spacing}{1.4}
  \begin{center}
    \textbf{ONLINE APPENDIX} \\
    \vspace{4mm}
    For the paper: \\
    \vspace{6mm}
           {\large \textsc{The impact of abortion legalization on fertility and maternal
               mortality: New evidence from Mexico}} \\
    \vspace{3mm}
    Damian Clarke and Hanna M\"uhlrad
\end{center}

\tableofcontents
\setlength\parindent{0.25in}
\setlength\parskip{0.25in}

\renewcommand{\theequation}{A\arabic{equation}}
\newpage

%%%%%%%%%%%%%%%%%%%%%%%%%%%%%%%%%%%%%%%%%%%%%%%%%%%%%%%%%%%%%%%%%%%%%%%%%%%%%%%%%%%%%%%%
%%%%%%%%% Data Appendix 
\section{Data Appendix}
\subsection{Vital Statistics Data}
Vital statistics data are publicly available from Mexico's National Institute of
Statistics and Geography (INEGI).  We download all microdata files from 2002 to 2014
(the most recent data available at the time of writing).  We then keep all records
corresponding to events occurring in 2002 to 2011.  We do not use data from 2012 onwards
given problems of reporting births in subsequent years, and concerns that births may be
only partially recorded until a number of years after the event \citep{CONAPO2012}. This
results in data on 23,157,119 births, and 11,858 cases of maternal deaths (see table
\ref{list_of_states}).

Maternal deaths are recorded officially in the vital statistics registers of deahts, and
are classified accourding to the 10\textsuperscript{th} revision of the International
Classification of Disease (ICD-10) codes.  Formally, maternal deaths are defined by the
WHO as ``The death of a woman while pregnant or within 42 days of termination of
pregnancy, irrespective of the duration and the site of the pregnancy, from any cause
related to or aggravated by the pregnancy or its management, but not from accidental or
incidental causes''.

The final sample consists of all births and maternal deaths of women aged 15-44 (most
fertile ages) occurring during 2002-2011, though results are not sensitive to including
all women up to the age of 49. The time period is chosen due to data availability and
quality. In the effort of meeting the UN Millennium Development Goals (MDG 5), a study
was conducted in 2002 in order to investigate misclassification of maternal mortality in
Mexico. Maternal deaths were found to be incorrectly registered in the death certificates
as well as underreported. This together with lagged reporting in live births (affecting
the denominator) gave the false impression of a reduction in the maternal mortality
ratio. To correct these deficiencies, large improvements were made in 2002 by
deliberately searching for maternal deaths as well as reclassifying maternal deaths
according to the standards of the WHO. As such, we begin analysis in year 2002, when the
updated maternal death reporting standards were put into place.

\subsection{Time Varying Controls}
We merge data on the female population with birth and death data aggregated at the level
of age, year and state.  This population data is obtained from the National Population
Council in Mexico (CONAPO), and is constructed from censal and inter-censal survey data,
which is a national representative sample of over ten million individuals stratified
geographically by municipality \citep{CONAPO2012}. This survey was conducted in 2000,
2005 and 2010 and the data between those years are interpolated using multiple imputation.

The economic and educational variables available at the state level are rates of
illiteracy, schooling and the proportion of the population that lacks health insurance.
These variables are obtained from the National Institute for Federalism and Municipal
Development (INAFED) and the National Education Statistical Information System (SNIE)
respectively. Information on the roll out of the national health insurance program
\emph{Seguro Popular} is obtained from the INEGI data bank.  The full dataset is made
available on the authors' webpages.

\subsection{MxFLS}\label{mxfls} 
In order to examine potential mechanisms through which the reform may have affected
fertility and maternal mortality, longitudinal data on contraceptive use and knowledge
is collected from the Mexican Family Life Survey (MxFLS). The MxFLS is a nationally
representative longitudinal dataset that follows individuals over time, covering various
topics regarding the well-being of the Mexican population including information on
reproductive health. This survey was conducted during three waves in 2002-2003, 2005-2006
and 2009-2012. The data between the years when the survey was conducted is generated by
linear interpolation. The analysis sample used consists of women aged 15-44 who completed the
reproductive health questionnaire resulting in a total of 15,114 individuals.
Further descriptive statistics are displayed in table \ref{MXFLS}.

In classifying contraceptive use, modern contraceptives are defined as condoms, oral
or injectable implants of hormones preventing ovulation, IUD, sterilization and
emergency contraception. Traditional or less safe methods are defined as the usage of
the calendar method or rhythm method, coitus interruptis, herbs or teas. A detailed
account of modern and traditional methods is provided by \citet{WHO2015b}.
 
 % Data on contraceptive use and knowledge among adult and teenage women are obtained from the Mexico National Health Survey (ENSA) conducted in 2000 and the Mexican National Survey of Health and Nutrition (ENSANUT) conducted in 2006 and 2012. This data is a nationally representative household survey of repeated cross sections, covering various topics regarding nutrition and health, including information on reproductive health. This survey was conducted in 2000, 2006 and 2012 and covers 45,870, 48,304 and 50,000 households per wave respectively. The data between the years when the survey was conducted is generated by linear interpolation. The sample used for constructing the state level contraceptive control variables, consists of women aged 15-44 who completed the reproductive health questionnaire.
 
\section{Additional Details of the ILE Reform}
\subsection{Reform Background}
A brief description of the reform and its timing is provided in section 2 of the main
paper.  In figure \ref{Map}, a map of Mexico is presented, with the reform are (Mexico
City) highlighted in grey.  In table \ref{Exposure}, we provide a break down of reform
timing and affected cohorts.  A medical abortion is a non-invasive procedure that causes
contractions of the womb terminating the pregnancy. The MOH-DF program offers both
surgical and medical abortion procedures and is the main provider of medical abortion
\citep{WinikoffSheldon2012}. Medical abortion procedures are safer and more
cost-efficient compared to other methods and the patient self-administers the abortion
at home. The large shift from 25\% of all abortion procedures being medical in 2007 to
as much as 74\% in 2011 have played a key part of meeting the demand \citep{Becker2013}.

A final concern regarding the identification strategy is interventions occurring at the
same time as the abortion reform. Of particular concern is the large scale welfare
program called Seguro Popular (SP) or ``People's' Insurance'' that was launched in 2002.
SP is a national health insurance program with the aim to provide the then 50 million
uninsured Mexican citizens with access to health care.  The evidence of SP's impact on
actual health outcomes is mixed. Results from multiple previous studies on SP show a
positive effect on public health care utilization (see for instance
\citet{knox_health_2008} and \cite{barros2008wealthier}). In contrast to these studies,
results indicating no effect on public health care utilization, medical expenditure and
self-assessed health has also been found \cite{king2009public}. Moreover, there is
evidence that SP is associated with lower infant mortality and improved antenatal care
\citep{conti2014evaluating} and better access to obstetric care
\citep{sosa2009heterogeneous}. However, to the best of our knowledge there is no
evidence of decreased maternal mortality, see for instance \citep{conti2014evaluating}
who analyze mortality. In 2005 all 32 states had enrolled in the SP program
\citep{Knauletal2007} and the program was implemented in a time varying fashion up until
2013.  As of 2010, 90\% of the uninsured population was affiliated with SP
\citep{bosch2012taking}. The SP program offers a wide range of health care services,
including 250 indispensable services and 57 expensive interventions of ``catastrophic
diseases'' \citep{Darney2015}. Amongst these are services related to childbirth,
antenatal care \citep{knox_health_2008} and contraceptives \citep{Darney2015}. However,
contraceptives and other family planning services have been provided free of charge by
the government since 1975 \citep{GIRE2009}. Moreover, to the best of our knowledge there
is no evidence of an effect on fertility from the SP program. Similarly, the effect on
mortality from the SP program has been documented only for infants \citep{conti2014evaluating}.
Nevertheless, the SP program may have a direct or indirect effect on fertility and
maternal mortality, which is why we control for the roll out of this program as well
as the percentage of population that is covered by health insurance at the state level.
 
\subsection{Spillover Effects}\label{Spillover}
The ILE reform took place in Mexico City, and in our main specification, we compare changes
in fertility and maternal deaths in Mexico City with those occurring in the rest of the
country. The causal effect of the reform can only be estimated in the absence of spillover
effects. Women with residency outside of Mexico City, traveling to Mexico City to have an
induced abortion would lead to a partially treated control group causing a downward bias in
the estimates.  Even in the presence of spillover effects however, the treatment effect can
be estimated if these spillover effects occur in areas located close to Mexico City but not
in more distant areas. This assumption is further supported by data available from the
public hospital records (MOH), showing that approximately 74\% of all women accessing the
reform are women with residency in Mexico City, 24\% are resident in the neighboring State
of Mexico and only 2\% of the women accessing the reform are residents in other states.
Accordingly, we assume that the spillover effects are concentrated to the State of Mexico.
In order to mitigate concern for spillover effects in the State of Mexico, which constitute
a large part of the greater metropolitan area of Mexico City, the State of Mexico is either
controlled for or omitted from the analysis. In addition, we estimate the treatment effect
using a specification to flexibly capture local spillover effects:
\begin{eqnarray}\label{eq3}
  \text{Outcome}_{ast}= \beta_{1} \text{ILE}_{st} + \sum_{j}\beta_{j}\text{Close}^j_{st} +
  X_{st}\delta +\alpha_{s} + \nu_{t} +\pi_{a}+ \lambda_{s}\times t +\varepsilon_{ast},
\end{eqnarray}
for $ \in \{0-5km, 6-10km, 11-15km, \ldots, 51-55km\}$.

This equation is analogous to equation 1 from the paper, however now including multiple
``close to treatment variables'' equal to one after the law was passed for states within
$j$ km of Mexico City. As in Equation 1, we include age, year and state fixed effects in
order to account for unobserved heterogeneity. In addition, we include state-specific
time trends as well as state level time-invariant controls. As in the main paper, the
fertility equation is estimated using OLS regression, and the maternal mortality
specification is estimated by Poisson regression.  Standard errors are clustered at the
level of the state.

Results to this specification are presented in figure \ref{SpilloverFigure}, which
find little evidence to suggest that spillover important spillover effects exist, at
least when agglomerating to the level of the state. 


%%%%%%%%%%%%%%%%%%%%%%%%%%%%%%%%%%%%%%%%%%%%%%%%%%%%%%%%%%%%%%%%%%%%%%%%%%%%%%%%%%%%%%%
\renewcommand{\thetable}{A\arabic{table}}
\setcounter{table}{0}
\renewcommand{\thefigure}{A\arabic{figure}}
\setcounter{figure}{0}


\section{Appendix Tables}
\begin{table}[H]
   \caption{List of States}\label{list_of_states}
   \begin{threeparttable}
     

\begin{tabular}{lcc}
	\hline\hline
&&\\
	\multirow{1}{*}{State} &
	\multicolumn{1}{c}{\shortstack{Number of \\ maternal deaths}}&\multicolumn{1}{c}{\shortstack{Number of \\ births}}\\ &&\\   \hline
 
Aguascalientes	&	87	&	257,576	\\
Baja California	&	250	&	5,68,276	\\
Baja California Sur	&	53	&	120,621	\\
Campeche	&	88	&	160,185	\\
Chiapas	&	825	&	1,220,254	\\
Chihuahua	&	440	&	678,600	\\
Coahuila	&	198	&	552,608	\\
Colima	&	31	&	120,840	\\
Distrito Federal (Mexico City)	&	818	&	1,505,790	\\
Durango	&	154	&	362,410	\\
Guanajuato	&	489	&	1,160,802	\\
Guerrero	&	702	&	809,783	\\
Hidalgo	&	297	&	538,216	\\
Jalisco	&	577	&	1,522,825	\\
State of México	&	1,745	&	3,186,751	\\
Michoacán	&	458	&	946,165	\\
Morelos	&	185	&	326,129	\\
Nayarit	&	109	&	216,272	\\
Nuevo León	&	204	&	882,618	\\
Oaxaca	&	639	&	851,138	\\
Puebla	&	739	&	1,377,091	\\
Querétaro	&	168	&	385,391	\\
Quintana Roo	&	136	&	256,223	\\
San Luis Potosí	&	276	&	552,094	\\
Sinaloa	&	173	&	554,838	\\
Sonora	&	197	&	513,172	\\
Tabasco	&	211	&	477,473	\\
Tamaulipas	&	258	&	630,260	\\
Tlaxcala	&	121	&	261,363	\\
Veracruz	&	922	&	1,467,936	\\
Yucatán	&	176	&	360,051	\\
Zacatecas	&	132	&	333,368	\\ 
 \hline\hline
\end{tabular}
           {\footnotesize
             \begin{tablenotes}
	     \item Administrative data made available by INEGI on all births and maternal deaths by each state for ages 15-44 during the time period 2002-2011.
           \end{tablenotes}}
   \end{threeparttable}
\end{table}
 
\begin{table}[H]
   \caption{Descriptives MxFLS data}\label{MXFLS}
   \begin{threeparttable}
     {\footnotesize
       

\begin{tabular}{lcccccc}
	\hline\hline
	&		&		&		&		&		&	 	\\
	\multirow{1}{*}{} &
	\multicolumn{3}{c}{Mexico City}&\multicolumn{3}{c}{The rest of Mexico}\\ \cmidrule(r){2-4} \cmidrule(l){5-7}


Variable	&	Mean	&	Std.	&	Obs	&	Mean	&	Std.	&	Obs	\\\hline


Contraception use	&	0,647	&	0,479	&	0,479	&	0,604	&	0,489	&	14766	\\
Modern Contraception use	&	0,612	&	0,488	&	0,488	&	0,563	&	0,496	&	14768	\\
Knowledge of contraception	&	0,981	&	0,137	&	0,137	&	0,969	&	0,173	&	23473	\\





\hline\hline
\end{tabular}}
     {\scriptsize 	
       \begin{tablenotes}
       \item Note to Table~\ref{MXFLS}: The data is obtained from the Mexican Family Life Survey (MxFLS).   
     \end{tablenotes}}  
   \end{threeparttable}
\end{table}


\begin{sidewaystable}[H]
  \caption{Exposure to the reform at time of conception}\label{Exposure}
  \centering
  \begin{threeparttable}  {\small \begin{tabular}{lccccccccc}
        \hline\hline   \multirow{1}{*}{} &
        \multicolumn{3}{c}{\textbf{Fertility}} & 	\multicolumn{3}{c}{\textbf{Maternal Mortality}} \\\cline{2-4}\cline{5-7} 
        \multirow{2}{*}{} &&
	\multicolumn{1}{c}{\textbf{Partial exposure}} & 	\multicolumn{1}{c}{\textbf{Full exposure}}&&\multicolumn{1}{c}{\textbf{Partial exposure}} & 	\multicolumn{1}{c}{\textbf{Full exposure}}\\ 
	\textit{Month of}   & \textit{Month of}  & \textit{Abortion legal at}   & \textit{Abortion legal at}  & \textit{Month of}  & \textit{Spillover effect}   & \textit{No spillover effects}\\ 
	\textit{Conception}& \textit{Birth}&\textit{at the month}&\textit{some point during} & \textit{Death}&\textit{from illegal  }&\textit{and abortion}\\
	&  &\textit{of conception}&\textit{the first trimester} & &\textit{abortions}&\textit{available the full}\\
	&  & & &  & &\textit{1st trimester}\\
	\hline 
        24 Dec 06-23 Jan 07	&	24 Sep-23 Oct, 2007	&	no	&	no	&	24 Apr-23 May, 2007	&	yes 	&	no	\\
        24 Jan-23 Feb, 2007	&	24 Oct-23 Nov, 2007	&	yes	&	no	&	24 May-23 Jun, 2007	&	yes 	&	no	\\
        24 Feb-23 Mar, 2007	&	24 Nov-23 Dec, 2007	&	yes	&	no	&	24 Jun-23 Jul, 2007	&	yes 	&	no	\\
        24 Mar-23 Apr, 2007	&	24 Dec 07-23 Jan, 08	&	yes	&	no	&	24 Jul-23 Aug, 2007	&	yes 	&	no	\\
        24 Apr-23 May, 2007	&	24 Jan-23 Feb, 2008	&	-	&	yes	&	24 Aug-23 Sep, 2007	&	yes 	&	no	\\
        24 May-23 Jun, 2007	&	24 Feb-23 Mar, 2008	&	-	&	yes	&	24 Sep-23 Oct, 2007	&	yes 	&	no	\\
        24 Jun-23 Jul, 2007	&	24 Mar-23 Apr, 2008	&	-	&	yes	&	24 Oct-23 Nov, 2007	&	no	&	yes	\\
        24 Jul-23 Aug, 2007	&	24 Apr-23 May, 2008	&	-	&	yes	&	24 Nov-23 Dec, 2007	&	-	&	yes	\\
        24 Aug-23 Sep, 2007	&	24 May-23 Jun, 2008	&	-	&	yes	&	24 Dec 07-23 Feb 08	&	-	&	yes	\\
        24 Sep-23 Oct, 2007	&	24 Jun-23 Jul, 2008	&	-	&	yes	&	24 Jan-23 Feb, 2008	&	-	&	yes	\\
        \hline 	\hline 
    \end{tabular} }
    \begin{tablenotes}
      \small This table displays exposure to the reform at time of conception for birth and maternal deaths respectively. The abortion law was passed on 24 April 2007. In terms of fertility outcome, the full effect of the reform will be observed in January 2008 since fertility exhibits a nine month lag due to realization of gestation (assuming 40 weeks of gestation). That is, women who got pregnant after the law was passed, had access to abortion during their whole first trimester and are therefore considered as fully treated. Women who got pregnant three months before the law was passed, had at least some period of time when they could access the reform, and are therefore considered partially treated. The earliest time for which any impact on fertility can be detected would then be from October 2007 and onwards. Regarding maternal mortality, women conceiving after 24 January 2007 could access the reform at least at some point during the first trimester while women conceiving after the reform was passed could access the reform during the whole first trimester. However, women conceiving before 24 January 2007, that is women pregnant beyond their first trimester at the time when the reform was passed, could have an illegal abortion during the remaining six months of their pregnancy and hence ``contaminate'' the maternal deaths taking place in the six months after the reform was passed. The fully exposed maternal deaths (without contamination) therefore occurs from 24 October and forward. Additionally, we also expect a lagged effect from maternal mortality since it is a function of fertility (i.e lower fertility implies lower mortality).  
    \end{tablenotes}
  \end{threeparttable} 	
\end{sidewaystable}  

\newgeometry{left=1.5cm, right=1.5cm,bottom=1.5cm}
\begin{table}[H]  \caption{Heterogeneous Effects of the Reform Across Ages} \label{heteroReg}
  \begin{threeparttable}
    \begin{subtable}{1\textwidth} \subcaption{Fertility}\label{heteroReg:1a}
      {\footnotesize  {
\def\sym#1{\ifmmode^{#1}\else\(^{#1}\)\fi}
\begin{tabular}{l*{6}{c}}
\hline\hline
                    &\multicolumn{1}{c}{\textbf{Ages 15-18}}&\multicolumn{1}{c}{\textbf{Ages 19-24}}&\multicolumn{1}{c}{\textbf{Ages 25-29}}&\multicolumn{1}{c}{\textbf{Ages 30-24}}&\multicolumn{1}{c}{\textbf{Ages 35-39}}&\multicolumn{1}{c}{\textbf{Ages 40-44}}\\\cmidrule(lr){2-2}\cmidrule(lr){3-3}\cmidrule(lr){4-4}\cmidrule(lr){5-5}\cmidrule(lr){6-6}\cmidrule(lr){7-7}
                    &\multicolumn{1}{c}{(1)}&\multicolumn{1}{c}{(2)}&\multicolumn{1}{c}{(3)}&\multicolumn{1}{c}{(4)}&\multicolumn{1}{c}{(5)}&\multicolumn{1}{c}{(6)}\\
                    &\multicolumn{1}{c}{\shortstack{OLS\\Log\\ Num. of \\Births}}&\multicolumn{1}{c}{\shortstack{OLS\\Log\\ Num. of \\Births}}&\multicolumn{1}{c}{\shortstack{OLS\\Log\\ Num. of \\Births}}&\multicolumn{1}{c}{\shortstack{OLS\\Log\\ Num. of \\Births}}&\multicolumn{1}{c}{\shortstack{OLS\\Log\\ Num. of \\Births}}&\multicolumn{1}{c}{\shortstack{OLS\\Log\\ Num. of \\Births}}\\
\hline
&&&&&&\\
Reform              &     -0.0872***&     -0.0124   &      0.0107   &     -0.0622***&     -0.0472***&     -0.0358** \\
                    &     (0.011)   &     (0.009)   &     (0.008)   &     (0.011)   &     (0.011)   &     (0.014)   \\
[1em]
ReformClose         &    -0.00861   &     0.00693   &      0.0128   &     0.00402   &     -0.0248** &      0.0282   \\
                    &     (0.013)   &     (0.014)   &     (0.012)   &     (0.015)   &     (0.010)   &     (0.020)   \\
\hline
\(R^{2}\)           &       0.994   &       0.998   &       0.998   &       0.997   &       0.994   &       0.987   \\
Mean\_of\_dep\_Var     &       7.753   &       8.644   &       8.495   &       8.091   &       7.292   &       5.731   \\
Observations (State*Year*Age) &1280 &1920&1600& 1600&1600&1600\\
Number of women &43106477 &60894528&47542195& 44176996&39516001&33952996\\
\hline FEs, Trends and Controls & \checkmark &\checkmark&\checkmark& \checkmark&\checkmark&\checkmark\\
 \bottomrule \bottomrule
\end{tabular}}
}
    \end{subtable}
    \begin{subtable}{1\textwidth}  \subcaption{Maternal Mortality}\label{heteroReg:1b}
      {\footnotesize  {
\def\sym#1{\ifmmode^{#1}\else\(^{#1}\)\fi}
\begin{tabular}{l*{6}{c}}
\hline\hline
                    &\multicolumn{1}{c}{\textbf{Ages 15-18}}&\multicolumn{1}{c}{\textbf{Ages 19-24}}&\multicolumn{1}{c}{\textbf{Ages 25-29}}&\multicolumn{1}{c}{\textbf{Ages 30-24}}&\multicolumn{1}{c}{\textbf{Ages 35-39}}&\multicolumn{1}{c}{\textbf{Ages 40-44}}\\\cmidrule(lr){2-2}\cmidrule(lr){3-3}\cmidrule(lr){4-4}\cmidrule(lr){5-5}\cmidrule(lr){6-6}\cmidrule(lr){7-7}
                    &\multicolumn{1}{c}{(1)}&\multicolumn{1}{c}{(2)}&\multicolumn{1}{c}{(3)}&\multicolumn{1}{c}{(4)}&\multicolumn{1}{c}{(5)}&\multicolumn{1}{c}{(6)}\\
                    &\multicolumn{1}{c}{\shortstack{Poisson\\\\ Num. of \\Deaths}}&\multicolumn{1}{c}{\shortstack{Poisson\\\\ Num. of \\Deaths}}&\multicolumn{1}{c}{\shortstack{Poisson\\\\ Num. of \\Deaths}}&\multicolumn{1}{c}{\shortstack{Poisson\\\\ Num. of \\Deaths}}&\multicolumn{1}{c}{\shortstack{Poisson\\\\ Num. of \\Deaths}}&\multicolumn{1}{c}{\shortstack{Poisson\\\\ Num. of \\Deaths}}\\
\hline
  &               &               &               &               &               &               \\
Reform              &      -0.593** &       0.308** &      -0.690***&      -0.393***&      0.0631   &      -0.371***\\
                    &     (0.253)   &     (0.137)   &     (0.105)   &     (0.128)   &     (0.152)   &     (0.125)   \\
[1em]
ReformClose         &      -0.460** &       0.129   &      0.0315   &       0.101   &       0.247   &     -0.0261   \\
                    &     (0.223)   &     (0.110)   &     (0.097)   &     (0.152)   &     (0.151)   &     (0.073)   \\
\hline
Pseudo \(R^{2}\)    &       0.313   &       0.331   &       0.330   &       0.332   &       0.293   &       0.228   \\
Mean\_of\_dep\_Var     &       1.727   &       2.731   &       2.791   &       2.861   &       2.354   &       1.037   \\
Observations (State*Year*Age) &1280 &1920&1600& 1600&1600&1600\\
Number of births &2732961 &8296491&5918229& 3948224&1817803&443411\\
\hline FEs, Trends and Controls & \checkmark &\checkmark&\checkmark& \checkmark&\checkmark&\checkmark\\
\bottomrule \bottomrule
\end{tabular}}
}
    \end{subtable}
    \begin{tablenotes}
      \footnotesize
    \item \textit{Notes to Table~\ref{heteroReg}}. The table shows the difference-in-difference estimates of the heterogeneous effects of the reform on (a) log number of births (b) and number of maternal deaths by age group. There are six different subsamples of the following age intervals: 15-18, 19-24, 25-29, 30-34, 35-39 and 40-44 for the period 2002-2011. The dependent variables are defined as the log number of births (\ref{heteroReg:1a}) and number of maternal deaths (\ref{heteroReg:1b}). The independent variable \textit{Reform} takes the value one in Mexico City nine months after the law was passed due to realization of gestation and zero otherwise. Similarly, \textit{ReformClose} is a binary variable equal to one in the neighboring State of Mexico nine months after the reform was introduced. The economic controls are state level income, unemployment and rurality. The controls for education are illiteracy and proportion of people aged 6-14 who are not enrolled in school. The health controls are the proportion of residents with no health rights and the roll out of the national health insurance program ``Seguro Popular''. The fixed effects consist of age, state and year binary variables. Standard errors are clustered at state level.*** p$<$0.01,** p$<$0.05,* p$<$0.1.
    \end{tablenotes}
  \end{threeparttable}
\end{table}
\restoregeometry

\begin{sidewaystable}[H]\caption{The Effect of the Reform on Births, monthly data} \label{BirthMonth}	
  \begin{threeparttable}
    {\footnotesize  {
\def\sym#1{\ifmmode^{#1}\else\(^{#1}\)\fi}
\begin{tabular}{l*{10}{c}}
\hline\hline
                    &\multicolumn{5}{c}{\textbf{Full-Sample 15-44}}                                 &\multicolumn{5}{c}{\textbf{Teenage women 15-19}}                               \\\cmidrule(lr){2-6}\cmidrule(lr){7-11}
                    &\multicolumn{1}{c}{(1)}&\multicolumn{1}{c}{(2)}&\multicolumn{1}{c}{(3)}&\multicolumn{1}{c}{(4)}&\multicolumn{1}{c}{(5)}&\multicolumn{1}{c}{(6)}&\multicolumn{1}{c}{(7)}&\multicolumn{1}{c}{(8)}&\multicolumn{1}{c}{(9)}&\multicolumn{1}{c}{(10)}\\
                    &\multicolumn{1}{c}{\shortstack{OLS\\Log\\ Num. of \\Births}}&\multicolumn{1}{c}{\shortstack{OLS\\Log \\Num. of \\Births}}&\multicolumn{1}{c}{\shortstack{OLS\\Log\\ Num. of \\Births}}&\multicolumn{1}{c}{\shortstack{OLS\\Log \\Num. of \\Births}}&\multicolumn{1}{c}{\shortstack{OLS\\Log\\ Num. of \\Births}}&\multicolumn{1}{c}{\shortstack{OLS\\Log\\ Num. of \\Births}}&\multicolumn{1}{c}{\shortstack{OLS\\Log \\Num. of \\Births}}&\multicolumn{1}{c}{\shortstack{OLS\\Log\\ Num. of \\Births}}&\multicolumn{1}{c}{\shortstack{OLS\\Log \\Num. of \\Births}}&\multicolumn{1}{c}{\shortstack{OLS\\Log\\ Num. of \\Births}}\\
\hline
PartiallyBirth      &     -0.0217*  &    -0.00821   &    -0.00692   &     -0.0101   &    -0.00919   &     -0.0153   &    -0.00489   &    -0.00284   &    -0.00526   &   -0.000861   \\
                    &     (0.013)   &     (0.009)   &     (0.009)   &     (0.011)   &     (0.011)   &     (0.014)   &     (0.011)   &     (0.010)   &     (0.011)   &     (0.012)   \\
[1em]
FullyBirth          &     -0.0307***&    -0.00781   &     -0.0166***&     -0.0147** &     -0.0144** &     -0.0551***&     -0.0375***&     -0.0517***&     -0.0461***&     -0.0467***\\
                    &     (0.011)   &     (0.005)   &     (0.006)   &     (0.006)   &     (0.006)   &     (0.011)   &     (0.007)   &     (0.008)   &     (0.009)   &     (0.008)   \\
\hline
\(R^{2}\)           &       0.970   &       0.970   &       0.970   &       0.970   &       0.970   &       0.977   &       0.978   &       0.978   &       0.978   &       0.978   \\
Mean\_of\_dep\_Var     &       4.562   &       4.562   &       4.562   &       4.562   &       4.562   &       4.846   &       4.846   &       4.846   &       4.846   &       4.846   \\
Observations (State*Year*Age) & 111529&111529& 111529&111529&111529&18653&18653&18653&18653&18653\\
\hline State, Year and Age FE& \checkmark &\checkmark&\checkmark& \checkmark&\checkmark&\checkmark&\checkmark&\checkmark&\checkmark&\checkmark\\
State Specific Linear Trends&&\checkmark&\checkmark&\checkmark&\checkmark&& \checkmark&\checkmark&\checkmark&\checkmark\\
Economic controls&& &\checkmark& \checkmark&\checkmark&&&\checkmark&\checkmark&\checkmark\\
Education controls&&&& \checkmark&\checkmark&&&&\checkmark&\checkmark\\
Health controls&&&&& \checkmark&&&&&\checkmark\\\bottomrule\bottomrule
\end{tabular}}
}
    \begin{tablenotes}
      \footnotesize
    \item \textit{Notes to Table~\ref{BirthMonth}}. The table displays the difference in difference estimates from OLS regressions. The sample consists of all births among women aged 15-44 (column 1-5) and sub-sample of teenage women aged 15-19 (column 6-10), for the period 2002-2011, on a monthly level. Partially treated women (who had some exposure to the reform but not during the entire trimester), such that births occurring between Oct-Dec 2007, are excluded from the sample. The dependent variables are defined as the log number of births. The independent variable \textit{Reform} takes the value one in Mexico City nine months after the law was passed due to realization of gestation and zero otherwise. Similarly, \textit{ReformClose} is a binary variable equal to one in the neighboring State of Mexico nine months after the reform was introduced. The economic controls are state level income, unemployment and rurality. The controls for education are illiteracy and proportion of people aged 6-14 who are not enrolled in school. The health controls are the proportion of residents with no health rights and the roll out of the national health insurance program ``Seguro Popular''. The fixed effects consist of age, state and year binary variables. Standard errors are clustered at state level.*** p$<$0.01,** p$<$0.05,* p$<$0.1.
    \end{tablenotes} 
  \end{threeparttable}
\end{sidewaystable}

\begin{sidewaystable}[H]\caption{The Effect of the Reform on Maternal Mortality, monthly data} \label{MMRMonth}
  \begin{threeparttable}
    {\footnotesize  {
\def\sym#1{\ifmmode^{#1}\else\(^{#1}\)\fi}
\begin{tabular}{l*{10}{c}}
\hline\hline
                    &\multicolumn{5}{c}{\textbf{Full-Sample 15-44}}                                 &\multicolumn{5}{c}{\textbf{Teenage women 15-19}}                               \\\cmidrule(lr){2-6}\cmidrule(lr){7-11}
                    &\multicolumn{1}{c}{(1)}&\multicolumn{1}{c}{(2)}&\multicolumn{1}{c}{(3)}&\multicolumn{1}{c}{(4)}&\multicolumn{1}{c}{(5)}&\multicolumn{1}{c}{(6)}&\multicolumn{1}{c}{(7)}&\multicolumn{1}{c}{(8)}&\multicolumn{1}{c}{(9)}&\multicolumn{1}{c}{(10)}\\
                    &\multicolumn{1}{c}{\shortstack{Poisson\\\\Num. \\Maternal \\Deaths}}&\multicolumn{1}{c}{\shortstack{Poisson\\\\Num. \\Maternal \\Deaths}}&\multicolumn{1}{c}{\shortstack{Poisson\\\\Num. \\Maternal \\Deaths}}&\multicolumn{1}{c}{\shortstack{Poisson\\\\Num. \\Maternal \\Deaths}}&\multicolumn{1}{c}{\shortstack{Poisson\\\\Num. \\Maternal \\Deaths}}&\multicolumn{1}{c}{\shortstack{Poisson\\\\Num. \\Maternal \\Deaths}}&\multicolumn{1}{c}{\shortstack{Poisson\\\\Num. \\Maternal \\Deaths}}&\multicolumn{1}{c}{\shortstack{Poisson\\\\Num. \\Maternal \\Deaths}}&\multicolumn{1}{c}{\shortstack{Poisson\\\\Num. \\Maternal \\Deaths}}&\multicolumn{1}{c}{\shortstack{Poisson\\\\Num. \\Maternal \\Deaths}}\\
\hline

PartiallyMMR        &      0.0947*  &      0.0744   &      0.0562   &      0.0608   &      0.0634   &       0.306*  &       0.230   &       0.229   &       0.211   &       0.212   \\
                    &     (0.052)   &     (0.066)   &     (0.065)   &     (0.064)   &     (0.066)   &     (0.179)   &     (0.172)   &     (0.176)   &     (0.174)   &     (0.175)   \\
[1em]
FullyMMR            &     -0.0414   &     -0.0728   &      -0.102** &     -0.0776*  &     -0.0809*  &     -0.0966   &      -0.212** &      -0.231** &      -0.331** &      -0.312*  \\
                    &     (0.026)   &     (0.049)   &     (0.048)   &     (0.046)   &     (0.045)   &     (0.061)   &     (0.089)   &     (0.100)   &     (0.147)   &     (0.169)   \\
\hline
Pseudo \(R^{2}\)    &     0.13641   &     0.13712   &     0.13725   &     0.13726   &     0.13726   &     0.14010   &     0.14307   &     0.14309   &     0.14325   &     0.14365   \\
Mean\_of\_dep\_Var     &       0.103   &       0.103   &       0.103   &       0.103   &       0.103   &      0.0785   &      0.0785   &      0.0785   &      0.0785   &      0.0785   \\
Observations (State*Year*Age) & 115146&115146& 115146&115146&115146&19259&19259&19259&19259&19259\\
\hline State, Year and Age FE& \checkmark &\checkmark&\checkmark& \checkmark&\checkmark&\checkmark&\checkmark&\checkmark&\checkmark&\checkmark\\
State Specific Linear Trends&&\checkmark&\checkmark&\checkmark&\checkmark&& \checkmark&\checkmark&\checkmark&\checkmark\\
Economic controls&& &\checkmark& \checkmark&\checkmark&&&\checkmark&\checkmark&\checkmark\\
Education controls&&&& \checkmark&\checkmark&&&&\checkmark&\checkmark\\
Health controls&&&&& \checkmark&&&&&\checkmark\\\bottomrule\bottomrule
\end{tabular}}
}
    \begin{tablenotes}
      \footnotesize
    \item \textit{Notes to Table~\ref{MMRMonth}}. The table displays the difference in difference estimates from OLS regressions. The sample consists of all maternal deaths among women aged 15-44 (column 1-5) and sub-sample of teenage women aged 15-19 (column 6-10), for the period 2002-2011, on a monthly level. Partially treated maternal deaths are deaths occurring between April-October 2007 because these deaths could be due to induced abortion beyond the first trimester, which would not occur if these women have had access to legal first trimester abortion. These are excluded from the sample. The dependent variables are defined as the number of maternal deaths. The independent variable \textit{Reform} takes the value one in Mexico City six months after the law was passed on order to account for late term abortions. Similarly, \textit{ReformClose} is a binary variable equal to one in the neighboring State of Mexico six months after the reform was introduced. The economic controls are state level income, unemployment and rurality. The controls for education are illiteracy and proportion of people aged 6-14 who are not enrolled in school. The health controls are the proportion of residents with no health rights and the roll out of the national health insurance program ``Seguro Popular''. The fixed effects consist of age, state and year binary variables. Standard errors are clustered at state level.*** p$<$0.01,** p$<$0.05,* p$<$0.1.
    \end{tablenotes} 
  \end{threeparttable}
\end{sidewaystable}


\begin{sidewaystable}[H]\centering  \caption{State of Mexico is Omitted} \label{robust_reg:a}
  \begin{threeparttable}
    {\footnotesize 	{
\def\sym#1{\ifmmode^{#1}\else\(^{#1}\)\fi}
\begin{tabular}{l*{10}{c}}
\hline\hline
                    &\multicolumn{5}{c}{\textbf{Fertility}}                                         &\multicolumn{5}{c}{\textbf{Maternal Mortality}}                                \\\cmidrule(lr){2-6}\cmidrule(lr){7-11}
                    &\multicolumn{1}{c}{(1)}&\multicolumn{1}{c}{(2)}&\multicolumn{1}{c}{(3)}&\multicolumn{1}{c}{(4)}&\multicolumn{1}{c}{(5)}&\multicolumn{1}{c}{(6)}&\multicolumn{1}{c}{(7)}&\multicolumn{1}{c}{(8)}&\multicolumn{1}{c}{(9)}&\multicolumn{1}{c}{(10)}\\
                    &\multicolumn{1}{c}{\shortstack{OLS\\Log\\ Num. of \\Births}}&\multicolumn{1}{c}{\shortstack{OLS\\Log \\Num. of \\Births}}&\multicolumn{1}{c}{\shortstack{OLS\\Log\\ Num. of \\Births}}&\multicolumn{1}{c}{\shortstack{OLS\\Log \\Num. of \\Births}}&\multicolumn{1}{c}{\shortstack{OLS\\Log\\ Num. of \\Births}}&\multicolumn{1}{c}{\shortstack{Poisson\\\\ Num. of \\Deaths}}&\multicolumn{1}{c}{\shortstack{Poisson\\\\ Num. of \\Deaths}}&\multicolumn{1}{c}{\shortstack{Poisson\\\\ Num. of \\Deaths}}&\multicolumn{1}{c}{\shortstack{Poisson\\\\ Num. of \\Deaths}}&\multicolumn{1}{c}{\shortstack{Poisson\\\\ Num. of \\Deaths}}\\
\hline
main                &               &               &               &               &               &               &               &               &               &               \\
Reform              &     -0.0217** &     -0.0285***&     -0.0374***&     -0.0338***&     -0.0341***&      -0.119***&      -0.216***&      -0.226***&      -0.193***&      -0.205***\\
                    &     (0.009)   &     (0.005)   &     (0.006)   &     (0.006)   &     (0.006)   &     (0.027)   &     (0.049)   &     (0.060)   &     (0.053)   &     (0.054)   \\
\hline
\(R^{2}\)           &       0.986   &       0.986   &       0.986   &       0.986   &       0.986   &               &               &               &               &               \\
Pseudo \(R^{2}\)    &               &               &               &               &               &       0.188   &       0.190   &       0.191   &       0.191   &       0.191   \\
Mean\_of\_dep\_Var     &       7.633   &       7.633   &       7.633   &       7.633   &       7.633   &       1.875   &       1.875   &       1.875   &       1.875   &       1.875   \\
Observations (State*Year*Age) & 9300&9300&9300&9300&9300&9300&9300&9300&9300&9300\\
Number of women & 231990879&231990879&231990879&231990879&231990879&&&&&\\
Number of births &&&&&& 19970368&19970368&19970368&19970368&19970368\\
\hline State, Year and Age FE& \checkmark &\checkmark&\checkmark& \checkmark&\checkmark&\checkmark&\checkmark&\checkmark&\checkmark&\checkmark\\
State Specific Linear Trends&&\checkmark&\checkmark&\checkmark&\checkmark&& \checkmark&\checkmark&\checkmark& \checkmark\\ 
Economic controls&& &\checkmark&\checkmark&\checkmark&& &\checkmark&\checkmark& \checkmark\\  
Education controls&& & &\checkmark&\checkmark&& & &\checkmark& \checkmark\\ 
Health controls&& &&&\checkmark&& &&& \checkmark\\ \bottomrule \bottomrule
\end{tabular}}
}
 
    
    \begin{tablenotes}  \footnotesize \item \textit{Note to Table~\ref{robust_reg:a}}. The table displays the difference in difference estimates from OLS regressions (column 1-5) and Poisson regressions (column 6-10). The sample consists of all births and maternal deaths among women aged 15-44 for the period 2002-2011. The State of Mexico is omitted due to potential spillover effects. The dependent variables are defined as the log number of births (column 1-5) and maternal deaths (column 6-10). The independent variable \textit{Reform} takes the value one in Mexico City nine months after the law was passed. Similarly, \textit{ReformClose} is a binary variable equal to one in the neighboring State of Mexico nine months after the reform was introduced. The economic controls are state level income, unemployment and rurality. The controls for education are state level illiteracy and proportion of people aged 6-14 who are not enrolled in school. The health controls are the proportion of residents with no health rights and the roll out of the national health insurance program ``Seguro Popular'' (state level). The fixed effects consist of age, state and year binary variables. Standard errors are clustered at state level.*** p$<$0.01,** p$<$0.05,* p$<$0.1.	 
    \end{tablenotes} 
  \end{threeparttable}
\end{sidewaystable}
 
 
 \begin{sidewaystable}[H]\centering  \caption{Robustness} \label{robust_reg:b}
 	\begin{threeparttable}
 	
  
 			{\footnotesize 	{
\def\sym#1{\ifmmode^{#1}\else\(^{#1}\)\fi}
\begin{tabular}{l*{10}{c}}
\hline\hline
                    &\multicolumn{5}{c}{\textbf{Fertility}}                                         &\multicolumn{5}{c}{\textbf{Maternal Mortality}}                                \\\cmidrule(lr){2-6}\cmidrule(lr){7-11}
                    &\multicolumn{1}{c}{(1)}&\multicolumn{1}{c}{(2)}&\multicolumn{1}{c}{(3)}&\multicolumn{1}{c}{(4)}&\multicolumn{1}{c}{(5)}&\multicolumn{1}{c}{(6)}&\multicolumn{1}{c}{(7)}&\multicolumn{1}{c}{(8)}&\multicolumn{1}{c}{(9)}&\multicolumn{1}{c}{(10)}\\
                    &\multicolumn{1}{c}{\shortstack{OLS\\Log\\ Num. of \\Births}}&\multicolumn{1}{c}{\shortstack{OLS\\Log \\Num. of \\Births}}&\multicolumn{1}{c}{\shortstack{OLS\\Log\\ Num. of \\Births}}&\multicolumn{1}{c}{\shortstack{OLS\\Log \\Num. of \\Births}}&\multicolumn{1}{c}{\shortstack{OLS\\Log\\ Num. of \\Births}}&\multicolumn{1}{c}{\shortstack{Poisson\\\\ Num. of \\Deaths}}&\multicolumn{1}{c}{\shortstack{Poisson\\\\ Num. of \\Deaths}}&\multicolumn{1}{c}{\shortstack{Poisson\\\\ Num. of \\Deaths}}&\multicolumn{1}{c}{\shortstack{Poisson\\\\ Num. of \\Deaths}}&\multicolumn{1}{c}{\shortstack{Poisson\\\\ Num. of \\Deaths}}\\
\hline
main                &               &               &               &               &               &               &               &               &               &               \\
Reform              &     -0.0283** &     -0.0261***&     -0.0308***&     -0.0331***&     -0.0330***&      -0.148***&      -0.217***&      -0.250***&      -0.278***&      -0.267***\\
                    &     (0.012)   &     (0.006)   &     (0.006)   &     (0.009)   &     (0.009)   &     (0.019)   &     (0.057)   &     (0.047)   &     (0.060)   &     (0.052)   \\
[1em]
ReformClose         &     -0.0149   &     0.00748   &     0.00536   &      0.0112   &      0.0129   &      -0.132***&     -0.0168   &      0.0176   &      0.0331   &      0.0632   \\
                    &     (0.012)   &     (0.006)   &     (0.006)   &     (0.013)   &     (0.016)   &     (0.019)   &     (0.057)   &     (0.058)   &     (0.071)   &     (0.071)   \\
\hline
\(R^{2}\)           &       0.990   &       0.990   &       0.990   &       0.990   &       0.990   &               &               &               &               &               \\
Pseudo \(R^{2}\)    &               &               &               &               &               &       0.371   &       0.372   &       0.372   &       0.372   &       0.372   \\
Mean\_of\_dep\_Var     &       7.888   &       7.888   &       7.888   &       7.888   &       7.888   &       2.823   &       2.823   &       2.823   &       2.823   &       2.823   \\
Observations (State*Year*Age) & 6300&6300&6300&6300&6300&6300&6300&6300&6300&6300\\
Number of women & 193099050&193099050&193099050&193099050&193099050&&&&&\\
Number of births &&&&&& 15763586&15763586&15763586&15763586&15763586\\
\hline State, Year and Age FE& \checkmark &\checkmark&\checkmark& \checkmark&\checkmark&\checkmark&\checkmark&\checkmark&\checkmark&\checkmark\\
State Specific Linear Trends&&\checkmark&\checkmark&\checkmark&\checkmark&& \checkmark&\checkmark&\checkmark& \checkmark\\ 
Economic controls&& &\checkmark&\checkmark&\checkmark&& &\checkmark&\checkmark& \checkmark\\  
Education controls&& & &\checkmark&\checkmark&& & &\checkmark& \checkmark\\ 
Health controls&& &&&\checkmark&& &&& \checkmark\\ \bottomrule \bottomrule
\end{tabular}}
}
 	 
 		
 		\begin{tablenotes} \footnotesize \item \textit{Note to Table~\ref{robust_reg:b}}. The table displays the difference in difference estimates from OLS regressions (column 1-5) and Poisson regressions (column 6-10). The sample consists of all births and maternal deaths among women aged 15-44 for the period 2002-2011. States where more than 50\% of all births occur in rural areas are omitted from the analysis. The dependent variables are defined as the log number of births (column 1-5) and maternal deaths (column 6-10). The independent variable \textit{Reform} takes the value one in Mexico City nine months after the law was passed. Similarly, \textit{ReformClose} is a binary variable equal to one in the neighboring State of Mexico after nine months after the reform was introduced. The economic controls are state level income, unemployment and rurality. The controls for education are state level illiteracy and proportion of people aged 6-14 who are not enrolled in school. The health controls are the proportion of residents with no health rights and the roll out of the national health insurance program ``Seguro Popular'' (state level). The fixed effects consist of age, state and year binary variables. Standard errors are clustered at state level.*** p$<$0.01,** p$<$0.05,* p$<$0.1.	 
 		\end{tablenotes} 
 	\end{threeparttable}
 \end{sidewaystable}
 

\begin{sidewaystable}[H]\centering \caption{Alternative Specifications}\label{altspecRates}
  \begin{threeparttable}
   
      {\footnotesize 	{
\def\sym#1{\ifmmode^{#1}\else\(^{#1}\)\fi}
\begin{tabular}{l*{10}{c}}
\hline\hline
                    &\multicolumn{5}{c}{\textbf{Fertility}}                                         &\multicolumn{5}{c}{\textbf{Maternal Mortality}}                                \\\cmidrule(lr){2-6}\cmidrule(lr){7-11}
                    &\multicolumn{1}{c}{(1)}&\multicolumn{1}{c}{(2)}&\multicolumn{1}{c}{(3)}&\multicolumn{1}{c}{(4)}&\multicolumn{1}{c}{(5)}&\multicolumn{1}{c}{(6)}&\multicolumn{1}{c}{(7)}&\multicolumn{1}{c}{(8)}&\multicolumn{1}{c}{(9)}&\multicolumn{1}{c}{(10)}\\
                    &\multicolumn{1}{c}{\shortstack{OLS\\\\ Birth Rate}}&\multicolumn{1}{c}{\shortstack{OLS\\\\ Birth Rate}}&\multicolumn{1}{c}{\shortstack{OLS\\\\ Birth Rate}}&\multicolumn{1}{c}{\shortstack{OLS\\\\ Birth Rate}}&\multicolumn{1}{c}{\shortstack{OLS\\\\ Birth Rate}}&\multicolumn{1}{c}{\shortstack{OLS\\\\ MMR}}&\multicolumn{1}{c}{\shortstack{OLS\\\\ MMR}}&\multicolumn{1}{c}{\shortstack{OLS\\\\ MMR}}&\multicolumn{1}{c}{\shortstack{OLS\\\\ MMR}}&\multicolumn{1}{c}{\shortstack{OLS\\\\ MMR}}\\
\hline
Reform              &       4.898***&      -1.758***&      -2.264***&      -2.337***&      -2.361***&      -3.061** &      -9.930***&      -11.21***&      -11.03***&      -11.32***\\
                    &     (0.705)   &     (0.382)   &     (0.535)   &     (0.584)   &     (0.585)   &     (1.295)   &     (1.753)   &     (2.202)   &     (2.225)   &     (2.097)   \\
[1em]
ReformClose         &      -1.094   &       0.843** &       0.489   &       0.701   &       0.624   &      -5.263***&       2.288   &       2.711   &       2.243   &       2.678   \\
                    &     (0.704)   &     (0.374)   &     (0.513)   &     (0.859)   &     (0.954)   &     (1.297)   &     (1.750)   &     (2.086)   &     (3.309)   &     (3.223)   \\
\hline
\(R^{2}\)           &       0.968   &       0.969   &       0.969   &       0.969   &       0.969   &       0.221   &       0.224   &       0.225   &       0.225   &       0.225   \\
Mean\_of\_dep\_Var     &       86.03   &       86.03   &       86.03   &       86.03   &       86.03   &       51.21   &       51.21   &       51.21   &       51.21   &       51.21   \\
Observations (State*Year*Age) & 9600&9600&9600&9600&9600&9600&9600&9600&9600&9600\\
Number of women & 269189193&269189193&269189193&269189193&269189193&&&&&\\
Number of births &&&&&& 23157119&23157119&23157119&23157119&23157119\\
\hline State, Year and Age FE& \checkmark &\checkmark&\checkmark& \checkmark&\checkmark&\checkmark&\checkmark&\checkmark&\checkmark&\checkmark\\
State Specific Linear Trends&&\checkmark&\checkmark&\checkmark&\checkmark&& \checkmark&\checkmark&\checkmark& \checkmark\\ 
Economic controls&& &\checkmark&\checkmark&\checkmark&& &\checkmark&\checkmark& \checkmark\\  
Education controls&& & &\checkmark&\checkmark&& & &\checkmark& \checkmark\\ 
Health controls&& &&&\checkmark&& &&& \checkmark\\ \bottomrule \bottomrule
\end{tabular}}
}
 

    

    \begin{tablenotes} 
      \footnotesize	\item \textit{Note to Table~\ref{altspecRates}}. The table displays the difference in difference estimates from OLS regressions (column 1-5) and Poisson regressions (column 6-10). The sample consists of all births and maternal deaths among women aged 15-44 for the period 2002-2011. The dependent variables are birth rate defined as the annual number of births per 1,000 women (column 1-5) in a specific area and maternal mortality ratio defined as the annual number of deaths per 100,000 live births (column 6-10). The independent variable \textit{Reform} takes the value one in Mexico City nine months after the law was passed. Similarly, \textit{ReformClose} is a binary variable equal to one in the neighboring State of Mexico after nine months after the reform was introduced. The economic controls are state level income, unemployment and rurality. The controls for education are state level illiteracy and proportion of people aged 6-14 who are not enrolled in school. The health controls are the proportion of residents with no health rights and the roll out of the national health insurance program ``Seguro Popular'' (state level). The fixed effects consist of age, state and year binary variables. Standard errors are clustered at state level.*** p$<$0.01,** p$<$0.05,* p$<$0.1.
      
       
    \end{tablenotes} 
  \end{threeparttable}
\end{sidewaystable} 



\begin{sidewaystable}[H]\centering \caption{Alternative Specifications}\label{OLS_levels}
	\begin{threeparttable}
	{ \footnotesize	{
\def\sym#1{\ifmmode^{#1}\else\(^{#1}\)\fi}
\begin{tabular}{l*{10}{c}}
\hline\hline
                    &\multicolumn{5}{c}{\textbf{Fertility}}                                         &\multicolumn{5}{c}{\textbf{Maternal Mortality}}                                \\\cmidrule(lr){2-6}\cmidrule(lr){7-11}
                    &\multicolumn{1}{c}{(1)}&\multicolumn{1}{c}{(2)}&\multicolumn{1}{c}{(3)}&\multicolumn{1}{c}{(4)}&\multicolumn{1}{c}{(5)}&\multicolumn{1}{c}{(6)}&\multicolumn{1}{c}{(7)}&\multicolumn{1}{c}{(8)}&\multicolumn{1}{c}{(9)}&\multicolumn{1}{c}{(10)}\\
                    &\multicolumn{1}{c}{\shortstack{OLS\\Num. of\\ Births}}&\multicolumn{1}{c}{\shortstack{OLS\\Num. of\\ Births}}&\multicolumn{1}{c}{\shortstack{OLS\\Num. of\\ Births}}&\multicolumn{1}{c}{\shortstack{OLS\\Num. of\\ Births}}&\multicolumn{1}{c}{\shortstack{OLS\\Num. of\\ Births}}&\multicolumn{1}{c}{\shortstack{OLS\\Num. of\\ Deaths}}&\multicolumn{1}{c}{\shortstack{OLS\\Num. of\\ Deaths}}&\multicolumn{1}{c}{\shortstack{OLS\\Num. of\\ Deaths}}&\multicolumn{1}{c}{\shortstack{OLS\\Num. of\\ Deaths}}&\multicolumn{1}{c}{\shortstack{OLS\\Num. of\\ Deaths}}\\
\hline
Reform              &      -442.7***&      -48.91** &      -59.30** &      -82.99*  &      -78.96*  &      -0.482***&      -0.558***&      -0.546***&      -0.579***&      -0.595***\\
                    &    (32.686)   &    (21.747)   &    (26.138)   &    (42.347)   &    (40.603)   &     (0.054)   &     (0.096)   &     (0.137)   &     (0.105)   &     (0.102)   \\
[1em]
ReformClose         &      -463.3***&       328.8***&       321.3***&       369.1***&       373.0***&      -1.098***&       0.347***&       0.392***&       0.467***&       0.499** \\
                    &    (32.238)   &    (21.013)   &    (24.272)   &    (65.804)   &    (70.545)   &     (0.052)   &     (0.095)   &     (0.125)   &     (0.169)   &     (0.183)   \\
\hline
\(R^{2}\)           &       0.815   &       0.816   &       0.816   &       0.816   &       0.816   &       0.612   &       0.618   &       0.618   &       0.618   &       0.619   \\
Mean\_of\_dep\_Var     &      4402.4   &      4402.4   &      4402.4   &      4402.4   &      4402.4   &       2.588   &       2.588   &       2.588   &       2.588   &       2.588   \\
Observations (State*Year*Age) & 9600&9600&9600&9600&9600&9600&9600&9600&9600&9600\\
Number of women & 269189193&269189193&269189193&269189193&269189193&&&&&\\
Number of births &&&&&& 23157119&23157119&23157119&23157119&23157119\\
\hline State, Year and Age FE& \checkmark &\checkmark&\checkmark& \checkmark&\checkmark&\checkmark&\checkmark&\checkmark&\checkmark&\checkmark\\
State Specific Linear Trends&&\checkmark&\checkmark&\checkmark&\checkmark&& \checkmark&\checkmark&\checkmark& \checkmark\\ 
Economic controls&& &\checkmark&\checkmark&\checkmark&& &\checkmark&\checkmark& \checkmark\\  
Education controls&& & &\checkmark&\checkmark&& & &\checkmark& \checkmark\\ 
Health controls&& &&&\checkmark&& &&& \checkmark\\ \bottomrule \bottomrule
\end{tabular}}
}
	 
		
	
		
		\begin{tablenotes} 
			\footnotesize
			\item \textit{Note to Table~\ref{OLS_levels}}. The table displays the difference in difference estimates from OLS regressions (column 1-10). The sample consists of all births and maternal deaths among women aged 15-44 for the period 2002-2011. The dependent variables are the numbers of births (column 1-5) and numbers of maternal deaths (column 6-10). The independent variable \textit{Reform} takes the value one in Mexico City nine months after the law was passed. Similarly, \textit{ReformClose} is a binary variable equal to one in the neighboring State of Mexico after nine months after the reform was introduced. The economic controls are state level income, unemployment and rurality. The controls for education are state level illiteracy and proportion of people aged 6-14 who are not enrolled in school. The health controls are the proportion of residents with no health rights and the roll out of the national health insurance program ``Seguro Popular'' (state level). The fixed effects consist of age, state and year binary variables. Standard errors are clustered at state level.*** p$<$0.01,** p$<$0.05,* p$<$0.1.
				
		\end{tablenotes} 
	\end{threeparttable}
\end{sidewaystable} 



\restoregeometry


\begin{sidewaystable}[H]\caption{Inference} \label{inference}
  \begin{threeparttable}
    {\footnotesize 	{
\def\sym#1{\ifmmode^{#1}\else\(^{#1}\)\fi}
\begin{tabular}{l*{10}{c}}
\hline\hline
                    &\multicolumn{5}{c}{\textbf{Fertility}}                                         &\multicolumn{5}{c}{\textbf{Maternal Mortality}}                                \\\cmidrule(lr){2-6}\cmidrule(lr){7-11}
                    &\multicolumn{1}{c}{(1)}&\multicolumn{1}{c}{(2)}&\multicolumn{1}{c}{(3)}&\multicolumn{1}{c}{(4)}&\multicolumn{1}{c}{(5)}&\multicolumn{1}{c}{(6)}&\multicolumn{1}{c}{(7)}&\multicolumn{1}{c}{(8)}&\multicolumn{1}{c}{(9)}&\multicolumn{1}{c}{(10)}\\
                    &\multicolumn{1}{c}{\shortstack{OLS\\Log Num \\ Births}}&\multicolumn{1}{c}{\shortstack{OLS\\Log Num \\ Births}}&\multicolumn{1}{c}{\shortstack{OLS\\Log Num \\ Births}}&\multicolumn{1}{c}{\shortstack{OLS\\Log Num \\ Births}}&\multicolumn{1}{c}{\shortstack{OLS\\Log Num \\ Births}}&\multicolumn{1}{c}{\shortstack{OLS\\Log Num \\ Deaths}}&\multicolumn{1}{c}{\shortstack{OLS\\Log Num \\ Deaths}}&\multicolumn{1}{c}{\shortstack{OLS\\Log Num \\ Deaths}}&\multicolumn{1}{c}{\shortstack{OLS\\Log Num \\ Deaths}}&\multicolumn{1}{c}{\shortstack{OLS\\Log Num \\ Deaths}}\\
\hline
Reform              &     -0.0300** &     -0.0283** &     -0.0370** &     -0.0385** &     -0.0382** &     -0.0674** &      -0.163** &      -0.185** &      -0.187** &      -0.189** \\
                    &     (0.013)   &     (0.012)   &     (0.016)   &     (0.017)   &     (0.016)   &     (0.029)   &     (0.070)   &     (0.080)   &     (0.080)   &     (0.081)   \\
[1em]
ReformClose         &     -0.0173   &     0.00290   &    -0.00258   &    -0.00292   &    -0.00766   &      -0.112** &       0.108***&       0.101***&      0.0911***&      0.0891** \\
                    &     (0.013)   &     (0.005)   &     (0.005)   &     (0.012)   &     (0.008)   &     (0.048)   &     (0.000)   &     (0.000)   &     (0.000)   &     (0.038)   \\
\hline
Observations        &        9600   &        9600   &        9600   &        9600   &        9600   &        9600   &        9600   &        9600   &        9600   &        9600   \\
\(R^{2}\)           &     0.98806   &     0.98853   &     0.98854   &     0.98854   &     0.98855   &     0.49129   &     0.49468   &     0.49508   &     0.49515   &     0.49520   \\

\hline State, Year and Age FE& \checkmark &\checkmark&\checkmark& \checkmark&\checkmark&\checkmark&\checkmark&\checkmark&\checkmark&\checkmark\\
State Specific Linear Trends&&\checkmark&\checkmark&\checkmark&\checkmark&& \checkmark&\checkmark&\checkmark& \checkmark\\ 
Economic controls&& &\checkmark&\checkmark&\checkmark&& &\checkmark&\checkmark& \checkmark\\  
Education controls&& & &\checkmark&\checkmark&& & &\checkmark& \checkmark\\ 
Health controls&& &&&\checkmark&& &&& \checkmark\\ \bottomrule \bottomrule
\end{tabular}}
}
    \begin{tablenotes}
      \footnotesize
    \item \textit{Note to Table~\ref{inference}}. The table displays the difference in difference estimates from OLS regressions (column 1-10). The sample consists of all births and maternal deaths among women aged 15-44 for the period 2002-2011. The dependent variables are the numbers of births (column 1-5) and numbers of maternal deaths (column 6-10) (wild bootstrapping is not compatible in non-linear models such as Poisson regressions). The independent variable \textit{Reform} takes the value one in Mexico City nine months after the law was passed. Similarly, \textit{ReformClose} is a binary variable equal to one in the neighboring State of Mexico after nine months after the reform was introduced. The economic controls are state level income, unemployment and rurality. The controls for education are state level illiteracy and proportion of people aged 6-14 who are not enrolled in school. The health controls are the proportion of residents with no health rights and the roll out of the national health insurance program ``Seguro Popular'' (state level). The fixed effects consist of age, state and year binary variables. standard errors are estimated using wild bootstrapping.*** p$<$0.01,** p$<$0.05,* p$<$0.1.
    
    \end{tablenotes} 
  \end{threeparttable}
\end{sidewaystable} 
\begin{table}[H]   \caption{Educational Composition} \label{educ_Birth_mmr}
  \begin{threeparttable}
    \begin{subtable}{\columnwidth} \centering \subcaption{Birth Register}\label{educ_b}
      {\small 	{
\def\sym#1{\ifmmode^{#1}\else\(^{#1}\)\fi}
\begin{tabular}{l*{5}{c}}
\hline\hline
                    &\multicolumn{5}{c}{\textbf{Maternal Composition, Education}}                   \\\cmidrule(lr){2-6}
                    &\multicolumn{1}{c}{(1)}&\multicolumn{1}{c}{(2)}&\multicolumn{1}{c}{(3)}&\multicolumn{1}{c}{(4)}&\multicolumn{1}{c}{(5)}\\
                    &\multicolumn{1}{c}{\shortstack{OLS\\No \\Education}}&\multicolumn{1}{c}{\shortstack{OLS\\Primary \\Education}}&\multicolumn{1}{c}{\shortstack{OLS\\Secondary \\Education}}&\multicolumn{1}{c}{\shortstack{OLS\\High \\School}}&\multicolumn{1}{c}{\shortstack{OLS\\Professional\\ Education}}\\
\hline
Reform              &    -0.00235   &     0.00417   &      0.0191***&     -0.0136***&    -0.00701***\\
                    &     (0.002)   &     (0.003)   &     (0.003)   &     (0.003)   &     (0.002)   \\
\hline
Observations        &     1081393   &     1081393   &     1081393   &     1081393   &     1081393   \\
\(R^{2}\)           &     0.10147   &     0.03111   &     0.05042   &     0.02922   &     0.05671   \\
Mean\_of\_dep\_Var     &       0.120   &       0.242   &       0.360   &       0.176   &       0.101   \\
\hline State and Year & \checkmark &\checkmark&\checkmark& \checkmark&\checkmark  \\
State Specific Linear Trends&\checkmark&\checkmark&\checkmark&\checkmark&\checkmark  \\ 
Economic controls&\checkmark&\checkmark &\checkmark&\checkmark&\checkmark \\
Health controls&\checkmark&\checkmark &\checkmark&\checkmark&\checkmark  \\ \bottomrule \bottomrule
\end{tabular}}
}
    \end{subtable}
    
    \begin{subtable}{\columnwidth} \centering \subcaption{Deaths Register}\label{educ_m}
      {\small 	{
\def\sym#1{\ifmmode^{#1}\else\(^{#1}\)\fi}
\begin{tabular}{l*{5}{c}}
\hline\hline
                    &\multicolumn{5}{c}{\textbf{Maternal Composition, Education}}                   \\\cmidrule(lr){2-6}
                    &\multicolumn{1}{c}{(1)}&\multicolumn{1}{c}{(2)}&\multicolumn{1}{c}{(3)}&\multicolumn{1}{c}{(4)}&\multicolumn{1}{c}{(5)}\\
                    &\multicolumn{1}{c}{\shortstack{OLS\\No \\Education}}&\multicolumn{1}{c}{\shortstack{OLS\\Primary \\Education}}&\multicolumn{1}{c}{\shortstack{OLS\\Secondary \\Education}}&\multicolumn{1}{c}{\shortstack{OLS\\High \\School}}&\multicolumn{1}{c}{\shortstack{OLS\\Professional\\ Education}}\\
\hline
Reform              &      0.0587** &     -0.0941***&      0.0296   &     -0.0247   &      0.0306** \\
                    &     (0.026)   &     (0.015)   &     (0.018)   &     (0.015)   &     (0.012)   \\
\hline
Observations        &       11556   &       11556   &       11556   &       11556   &       11556   \\
\(R^{2}\)           &     0.12714   &     0.08852   &     0.05357   &     0.04339   &     0.02780   \\
Mean\_of\_dep\_Var     &       0.443   &       0.101   &       0.245   &       0.126   &      0.0852   \\
\hline State and Year & \checkmark &\checkmark&\checkmark& \checkmark&\checkmark  \\
State Specific Linear Trends&\checkmark&\checkmark&\checkmark&\checkmark&\checkmark  \\ 
Economic controls&\checkmark&\checkmark &\checkmark&\checkmark&\checkmark \\
Health controls&\checkmark&\checkmark &\checkmark&\checkmark&\checkmark  \\ \bottomrule \bottomrule
\end{tabular}}
}
    \end{subtable}
    
    \begin{tablenotes} 
      \footnotesize	\item \textit{Note to Table~\ref{educ_Birth_mmr}.} The table displays the difference in difference estimates from OLS regressions. The sample consists of all births and maternal deaths for women aged 15-44 for the period 2002-2011. The dependent variables are, conditional on motherhood, the probability of having no or incomplete primary education (column 1), probability of primary education (column 2), probability of secondary education (column 3), probability of high school degree (column 4), probability of professional education (column 5) age (column 6) teenage death (column 7) and probability of single motherhood. The independent variable \textit{Reform} takes the value of one in locations and time where the reform was passed i.e. in the Mexico City after 2007. A full set of fixed effects, time varying controls and state-specific linear time trends are included in each regression. Standard errors are clustered at state level.*** p$<$0.01,** p$<$0.05,* p$<$0.1.  
    \end{tablenotes}
  \end{threeparttable}
\end{table}


%%%%%%%%%%%%%%%%%%%%%%%%%%%%%%%%%%%%%%%%%%%%%%%%%%%%%%%%%%%%%%%%%%%%%%%%%%%%%%%%
\section{Appendix Figures}
\begin{figure}[H]
  \caption{Map of Mexico}\label{Map}\centering
  \includegraphics[scale=0.6]{figures/MapMexico.pdf}
\end{figure}
 
\begin{figure}[H]
  \centering	\caption{MxFLS data}
  \label{MxFLS_Graphs}
  \begin{subfigure}{.3\textwidth}
    \centering	\caption{Current use of modern method}\label{modmethod}
    \includegraphics[scale=0.3]{figures/Trend_ModMethod.pdf}
  \end{subfigure}%
  \begin{subfigure}{.3\textwidth}
    \centering\caption{Current use of modern or traditional method}\label{anymethod}
    \includegraphics[scale=0.3]{figures/Trend_AnyMethod.pdf}
  \end{subfigure}%
  \begin{subfigure}{.3\textwidth}
    \centering	\caption{Contraceptive knowledge}	\label{knowledge}
    \includegraphics[scale=0.3]{figures/Trend_ContraKnow.pdf}
  \end{subfigure}
      {\footnotesize    	\floatfoot{Note to Figure~\ref{MxFLS_Graphs}: The data is obtained from the Mexican Family Life Survey (MxFLS) conducted in 2002-2003, 2005-2006 and 2009-2012. The sample consists of all women aged 15-44 who completed the reproductive health questionnaire resulting in a total of 15,114 individuals. The data for the years between when the surveys were conducted is linearly interpolated. Modern contraceptives are condoms, oral or/injectable/implants of hormones preventing ovulation, IUD, sterilization and emergency contraception. Traditional or less safe methods are calendar method or rhythm method, coitus interrupts, herbs or teas. }}
\end{figure}

\begin{figure}[H]
  \centering\caption{Spillover Effects}\label{SpilloverFigure}
  \begin{subfigure}{0.5\textwidth}
    \centering\caption{Fertility}\label{spillover_birth}
    \includegraphics[scale=0.55]{figures/SpillOverBirth.pdf}
  \end{subfigure}%
  \begin{subfigure}{0.5 \textwidth}
    \centering\caption{Maternal mortality}\label{spillover_MMR}
    \includegraphics[scale=0.55]{figures/SpillOverMMR.pdf}
  \end{subfigure}
  \floatfoot{Note to figure ~\ref{SpilloverFigure}: The coefficient plot shows point estimates and confidence intervals estimated according to Equation~\eqref{eq3} using OLS regressions (Figure~\ref*{spillover_birth}) and Poisson regressions (Figure~\ref*{spillover_MMR}). The sample consists of all births and maternal deaths among women aged 15-44 for the period 2002-2011. We include multiple treatment variables equal to unity (9 months) after the law was passed for states with the minimum distance intervals 0-5, 6-10, 11-15 up to 51-55 kilometers to Mexico City, in order to examine spillover effects. DF=Mexico City, Mex=State of Mexico, MR=Morelos, HI=Hidalgo, PU=Puebla and TL=Tlaxcala. A full set of fixed effects, time varying controls and state-specific linear time trends are included in each regression (see note to Table~\ref{heteroReg}).}
\end{figure}

 

\newpage
\bibliographystyle{ecta}
\bibliography{refs}


\end{spacing}
\end{document}

