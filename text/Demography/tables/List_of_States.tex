

\begin{tabular}{lcc}
	\hline\hline
&&\\
	\multirow{1}{*}{State} &
	\multicolumn{1}{c}{\shortstack{Number of \\ maternal deaths}}&\multicolumn{1}{c}{\shortstack{Number of \\ births}}\\ &&\\   \hline
 
Aguascalientes	&	87	&	257,576	\\
Baja California	&	250	&	5,68,276	\\
Baja California Sur	&	53	&	120,621	\\
Campeche	&	88	&	160,185	\\
Chiapas	&	825	&	1,220,254	\\
Chihuahua	&	440	&	678,600	\\
Coahuila	&	198	&	552,608	\\
Colima	&	31	&	120,840	\\
Distrito Federal (Mexico City)	&	818	&	1,505,790	\\
Durango	&	154	&	362,410	\\
Guanajuato	&	489	&	1,160,802	\\
Guerrero	&	702	&	809,783	\\
Hidalgo	&	297	&	538,216	\\
Jalisco	&	577	&	1,522,825	\\
State of México	&	1,745	&	3,186,751	\\
Michoacán	&	458	&	946,165	\\
Morelos	&	185	&	326,129	\\
Nayarit	&	109	&	216,272	\\
Nuevo León	&	204	&	882,618	\\
Oaxaca	&	639	&	851,138	\\
Puebla	&	739	&	1,377,091	\\
Querétaro	&	168	&	385,391	\\
Quintana Roo	&	136	&	256,223	\\
San Luis Potosí	&	276	&	552,094	\\
Sinaloa	&	173	&	554,838	\\
Sonora	&	197	&	513,172	\\
Tabasco	&	211	&	477,473	\\
Tamaulipas	&	258	&	630,260	\\
Tlaxcala	&	121	&	261,363	\\
Veracruz	&	922	&	1,467,936	\\
Yucatán	&	176	&	360,051	\\
Zacatecas	&	132	&	333,368	\\ 
 \hline\hline
\end{tabular}